\documentclass[English, 12pt]{article}
\usepackage{amsmath}
\usepackage{amssymb}
\usepackage{pgfplots}
\title{Complex Practice}
\author{Sam Kantor}

\newtheorem{theorem}{Theorem}[section]
\newtheorem{lemma}[theorem]{Lemma}
\newtheorem{proposition}[theorem]{Proposition}
\newtheorem{corollary}[theorem]{Corollary}
\newtheorem{example}[theorem]{Example}



\begin{document}
\maketitle
\setcounter{section}{38}
\begin{proposition}
If $Z_1 = R_1(\cos (\theta) + i\sin(\theta))$ and $Z_1 = R_1(\cos (\theta_2) + i\sin(\theta_2))$ which are two complex numbers in polar form, then $$Z_1 \cdot Z_2 = R_1 \cdot R_2 (\cos(\theta_1 + \theta_2) + \sin(\theta_1 + \theta_2)))$$
\end{proposition}

\begin{example}
From 37.4
\begin{align*}
 & (\sqrt{6} + \sqrt{2i}) \cdot (-3\cdot\sqrt[•]{2} + 3\cdot\sqrt[•]{6i}) \\
 &=(2\cdot\sqrt[•]{2}\cdot(\cos \frac{\pi}{6} + i\sin \frac{\pi}{6}))(6\cdot\sqrt[•]{2}\cdot(\cos \frac{2\pi}{3} + i\sin \frac{2\pi}{3})) \\
 &= (2\cdot \sqrt[•]{2}) (6\cdot \sqrt[•]{2})(\cos (\frac{\pi}{6} + \frac{2\pi}{3}) + i\sin (\frac{\pi}{6} + \frac{2 \pi}{3}) \\
 &= (24)(\cos (\frac{5\pi}{6}) + i\sin (\frac{5\pi}{6})) \\
 &= -12\cdot \sqrt[•]{3} + 12i   
\end{align*}
\end{example}

\begin{example} 
If $Z = r(cos(\theta + i\sin(\theta)))$	 \\
Calculate IZ. 
\begin{align*}
IZ &= (\cos\frac{\pi}{2} + \sin\frac{\pi}{2}) (r (\cos\theta + \sin\theta) \\
   &= r(\cos(\theta + \frac{\pi}{2}) + i\sin(\theta + \frac{\pi}{2})
\end{align*}
\newpage
\Large{Proof of PCMN}
\begin{align*}
Z_1 \cdot Z_2 &= (r_1 (\cos \theta_1 + i\sin \theta_1)) (r_2 (\cos \theta_2 + i\sin \theta_2)) \\
&= (r_1\cdot r_2) (\cos \theta_1 + i\sin\theta_1) (\cos\theta_2 + i\sin\theta_2)\\
&= (r_1 \cdot r_2) (\cos (\theta_1 + \theta_2)) + (i\sin (\theta_1 + \theta_2))
\end{align*}

\end{example}
\begin{theorem} Demoives Theorem (DMT) \\
If $\theta \in \mathbb{R}, n \in \mathbb{Z}$ Then 
\begin{equation*}
(\cos\theta + i\sin\theta)^n = \cos (n\cdot\theta) + i\sin(n\cdot \theta)
\end{equation*}
\end{theorem}

\begin{corollary} .\\
IF
\begin{equation*}
Z = R(cos\theta + isin(\theta)
\end{equation*}
Then
\begin{equation*}
Z = R(cos(n\cdot \theta) + isin(n\cdot \theta))
\end{equation*}
\end{corollary}

\begin{example} Evaluate $ (1-\frac{1}{\sqrt{3}}i)^{10} $
\begin{align*}
& 1-\frac{1}{\sqrt{3}} = \cos (0) \therefore \theta = 0 \\
& \frac{-1}{\sqrt{3}} = \sin (\frac{-\pi}{3}) \therefore \theta = \frac{-\pi}{3}
\end{align*}
$(\cos(0) + i\sin (\frac{-\pi}{3}))^{10}$ \\
\begin{align*}
&= \cos(10 \cdot 0) + i\sin (10 \cdot \frac{-\pi}{3})\\
&= 1 + i\sin (\frac{-10\pi}{3}) \\
&= 1 + i\sin (\frac{2\pi}{3}) \\
&= 1 + i \frac{\sqrt{3}}{2} \\
\end{align*}

\end{example}

\end{document}