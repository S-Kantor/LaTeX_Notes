\documentclass[12pt]{article}
\usepackage{amsmath}
\usepackage{amssymb}
\title{Introduction to Vectors}
\author{Sam Kantor \\ Instructor Dan Wolczuk}

\newtheorem{theorem}{Theorem}[section]
\newtheorem{lemma}[theorem]{Lemma}
\newtheorem{proposition}[theorem]{Proposition}
\newtheorem{corollary}[theorem]{Corollary}
\newtheorem{example}[theorem]{Example}



\begin{document}

\maketitle
\newpage

\Large {Vectors in $\mathbb{R}^N $}
\\
\begin{theorem}
$$  \mathbb{R}^n =
 {\left[\begin{matrix}
    X_i \\ \vdots \\
    X_n
\end{matrix}\right]} \, | \: X_i, \, \dots X_n \in \mathbb{R} $$

\end{theorem}

\Large {Sometimes to better understand vectors in $\mathbb{R}^n$, we can use points to visualize them.} \\

$\vec{x} = {\left[\begin{matrix}
    X_i \\ \vdots \\
    X_n
\end{matrix}\right]}$ is equivalent to ($x_i$, $\dots$, $x_n$)

\begin{example}

Vector $\vec{x} = {\left[\begin{matrix}
    1 \\ 2
\end{matrix}\right]}$ as point $(1,2)$

\end{example}

\Large {Vectors always start at the origin and end at the points it's defined by.}

\begin{theorem}
The set of vectors of the form $$\vec{x} = {\left[\begin{matrix}
    s \\ t \\ 0
\end{matrix}\right]}, s, t \in \mathbb{R}$$ are the set of all points of the form $(s, t, 0)$. Hence the plane in $\mathbb{R}^3$
\end{theorem}

\textbf{Definition}
Let $\vec{x} = {\left[\begin{matrix}
    X_i \\ \vdots \\ X_n
\end{matrix}\right]}, \vec{y} = {\left[\begin{matrix}
    Y_i \\ \vdots \\ Y_n
\end{matrix}\right]} \in \mathbb{R}$, we say that $\vec{x}$ and $\vec{y}$ are \textbf{equal} and write $\vec{x} = \vec{y}$ \, if $x_i = y_i$ for all $i <= x <= y$. \\

\textbf{Definition} \\
We define addition by $\vec{x}$ + $\vec{y}$ \: = ${\left[\begin{matrix}
    X_i &+ &Y_1 \\ & \vdots \\ X_n &+ &Y_n
\end{matrix}\right]}$ \\

For any real scalar $t \in \mathbb{R}$, we define scalar \\ multiplication by: $$t\vec{x} = {\left[\begin{matrix}
    tx_i\\ \vdots \\ tx_n
\end{matrix}\right]}$$

\begin{example} - \\
${\left[\begin{matrix}
   1 \\ 2
\end{matrix}\right]}$ +
${\left[\begin{matrix}
   2 \\ -3
\end{matrix}\right]}$ = 
${\left[\begin{matrix}
   3 \\ -1
\end{matrix}\right]}$
\end{example}

\begin{example} - \\
$3\cdot{\left[\begin{matrix}
   2 \\ 4
\end{matrix}\right]}$ = 
${\left[\begin{matrix}
   6 \\ 12
\end{matrix}\right]}$
\end{example}

\begin{example} - \\
$(-1)\cdot{\left[\begin{matrix}
   2 \\ 5
\end{matrix}\right]}$ +
$(\sqrt{2})\cdot {\left[\begin{matrix}
   2 \\ \frac{1}{\sqrt{2}}
\end{matrix}\right]}$ = 
${\left[\begin{matrix}
   tbd \\ tbd
\end{matrix}\right]}$
\end{example}

\end{document}